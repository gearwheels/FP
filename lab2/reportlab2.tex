\documentclass[12pt]{article}

\usepackage{fullpage}
\usepackage{multicol,multirow}
\usepackage{tabularx}
\usepackage{ulem}
\usepackage[utf8]{inputenc}
\usepackage[russian]{babel}
\usepackage{amsmath}
\usepackage{amssymb}

\usepackage{titlesec}

\titleformat{\section}
  {\normalfont\Large\bfseries}{\thesection.}{0.3em}{}

\titleformat{\subsection}
  {\normalfont\large\bfseries}{\thesubsection.}{0.3em}{}

\titlespacing{\section}{0pt}{*2}{*2}
\titlespacing{\subsection}{0pt}{*1}{*1}
\titlespacing{\subsubsection}{0pt}{*0}{*0}
\usepackage{listings}
\lstloadlanguages{Lisp}
\lstset{extendedchars=false,
	breaklines=true,
	breakatwhitespace=true,
	keepspaces = true,
	tabsize=2
}
\begin{document}

\section*{Отчет по лабораторной работе № 2 \\
по курсу \guillemotleft Функциональное программирование\guillemotright}
\begin{flushright}
Студент группы М8О-307-19 МАИ \textit{Тимофеев Алексей Владимирович}, \textnumero 21 по списку \\
\makebox[7cm]{Контакты: {\tt TimofeevAV8f@yandex.ru} \hfill} \\
\makebox[7cm]{Работа выполнена: 13.04.2022 \hfill} \\
\ \\
Преподаватель: Иванов Дмитрий Анатольевич, доц. каф. 806 \\
\makebox[7cm]{Отчет сдан: \hfill} \\
\makebox[7cm]{Итоговая оценка: \hfill} \\
\makebox[7cm]{Подпись преподавателя: \hfill} \\

\end{flushright}

\section{Тема работы}
Простейшие функции работы со списками Коммон Лисп.

\section{Цель работы}
Научиться конструировать списки, находить элемент в списке, использовать схему линейной и древовидной рекурсии для обхода и реконструкции плоских списков и деревьев.

\section{Задание (Вариант 2.39)}
Запрограммируйте рекурсивно на языке Коммон Лисп функцию набор (х), принимающую один аргумент - список х с подсписками любой глубины.\\

Функция должна возвращать список атомов (отличных от NIL), обладающий тем свойством, что все атомы, появляющиеся в х, появляются в результирующем списке в том же самом порядке.\\


\section{Оборудование студента}
Процессор Intel Core i5-10600K @ 4.10GHz, память: 16 Gb, разрядность системы: 64.

\section{Программное обеспечение}
ОС Ubuntu 20.04.4 LTS, комилятор GNU CLISP 2.49.92, текстовый редактор VS Code

\pagebreak
\section{Идея, метод, алгоритм}
Функция получает список списков, далее идет проверка является ли список пустым, если нет проверяется атом это или список. Если атом он включается в результирующий список, если это список то рекурсивно доходим до его атомов.

\section{Сценарий выполнения работы}

\section{Распечатка программы и её результаты}

\subsection{Исходный код}
\lstinputlisting{./laba2.lisp}

\pagebreak
\subsection{Результаты работы}
\lstinputlisting{./log2.txt}

\pagebreak
\section{Дневник отладки}
\begin{tabular}{|p{50pt}|p{80pt}|p{140pt}|p{140pt}|}
\hline
Дата & Событие & Действие по исправлению & Примечание \\
\hline
\end{tabular}

\section{Замечания автора по существу работы}
В задании не оганичена длинна списка, поэтому если задать список из очень большого количества элементов, может произойти переполнение стека, так как программа завязана на рекурсии. С большей вероятностью это может проявиться на слабых системах (слабых в смысле железа).


\section{Выводы}
При выполнении лабораторной работы № 2 я познакомился со списками языка Common Lisp. Также я потрентровался писать рекурсию. Реализация рекурсии в Common Lisp намного проще чем в привычных нам Си образных языках.




\end{document}

\documentclass[12pt]{article}

\usepackage{fullpage}
\usepackage{multicol,multirow}
\usepackage{tabularx}
\usepackage{ulem}
\usepackage[utf8]{inputenc}
\usepackage[russian]{babel}
\usepackage{amsmath}
\usepackage{amssymb}

\usepackage{titlesec}

\titleformat{\section}
  {\normalfont\Large\bfseries}{\thesection.}{0.3em}{}

\titleformat{\subsection}
  {\normalfont\large\bfseries}{\thesubsection.}{0.3em}{}

\titlespacing{\section}{0pt}{*2}{*2}
\titlespacing{\subsection}{0pt}{*1}{*1}
\titlespacing{\subsubsection}{0pt}{*0}{*0}
\usepackage{listings}
\lstloadlanguages{Lisp}
\lstset{extendedchars=false,
	breaklines=true,
	breakatwhitespace=true,
	keepspaces = true,
	tabsize=2
}
\begin{document}

\section*{Отчет по лабораторной работе № 3 \\
по курсу \guillemotleft Функциональное программирование\guillemotright}
\begin{flushright}
Студент группы М8О-307-19 МАИ \textit{Тимофеев Алексей Владимирович}, \textnumero 21 по списку \\
\makebox[7cm]{Контакты: {\tt TimofeevAV8f@yandex.ru} \hfill} \\
\makebox[7cm]{Работа выполнена: 27.04.2022 \hfill} \\
\ \\
Преподаватель: Иванов Дмитрий Анатольевич, доц. каф. 806 \\
\makebox[7cm]{Отчет сдан: \hfill} \\
\makebox[7cm]{Итоговая оценка: \hfill} \\
\makebox[7cm]{Подпись преподавателя: \hfill} \\

\end{flushright}

\section{Тема работы}
Последовательности, массивы и управляющие конструкции Коммон Лисп.

\section{Цель работы}
Научиться создавать векторы и массивы для представления матриц, освоить общие функции работы с последовательностями, инструкции цикла и нелокального выхода.

\section{Задание (Вариант 3.18)}

Запрограммировать на языке Коммон Лисп функцию, принимающую три аргумента:\\
A - двумерный массив, представляющий действительную матрицу размера m×n,\\
v - вектор действительных чисел длины m,\\
j - номер столбца, 0 \leq j \leq n.\\

Функция должна возвращать новую матрицу размера m×(n+1), полученную вставкой после столбца с номером j нового столбца с элементами из v. j=0 означает вставку перед самым первым столбцом.\\

Исходный массив A должен оставаться неизменным.\\


\section{Оборудование студента}
Процессор Intel Core i5-10600K @ 4.10GHz, память: 16 Gb, разрядность системы: 64.

\section{Программное обеспечение}
ОС Ubuntu 20.04.4 LTS, комилятор GNU CLISP 2.49.92, текстовый редактор VS Code

\pagebreak
\section{Идея, метод, алгоритм}
Так как в задании сказано только про реализацию функции обработки матриц, но не сказано про ввод-вывод, я решил заранее заготовить тесты и проинициализировать их в файле программы для удобства отладки.

Программа состоит из 3-х функций:\\
mainFun - точка входа.\\
extend-matrix - основная функция обработки матрицы. \\
print-matrix - функция печати матрицы.\\

Суть extend-matrix такова, первым аргументом в функцию идет исходная матрица $a$, вторым число - номер столбца $k$, перед которым нужно вставить новый столбец, который передается в третьем аргументе функции $v$. Сначала создается новая матрица $b$ размерности $m×(n+1)$, далее она инициализируется значениями из м-цы $a$ до $k$-го столбца, затем инициализируется $k$-ый столбец значениями из $v$, в конце матрица $b$ дозаполняется оставшимися значениями из $a$. Далее она передается на печать.

\section{Сценарий выполнения работы}

\section{Распечатка программы и её результаты}

\subsection{Исходный код}
\lstinputlisting{./lab3.lisp}

\pagebreak
\subsection{Результаты работы}
\lstinputlisting{./log3.txt}

\pagebreak
\section{Дневник отладки}
\begin{tabular}{|p{50pt}|p{80pt}|p{140pt}|p{140pt}|}
\hline
Дата & Событие & Действие по исправлению & Примечание \\
1 мая 2022 & Ошибка при подаче в функцию extend-matrix вектора для вставки & Добавил считывание длины вектора, исправлен цикл обработки вектора & Невнимательно прочитал задание и сделал так, что в функцию extend-matrix поступает вместо вектора список\\
\hline
\end{tabular}

\section{Замечания автора по существу работы}
В задании сказано, что нужно реализовать только функцию обработки матриц, но ничего не сказано про функции ввода-вывода матриц, поэтому я написал только функцию вывода для удобства отладки.


\section{Выводы}
При выполнении лабораторной работы № 3 я познакомился со встроенными функциями и инструментами, а также управляющими конструкциями коммон лисп с помощью которых
мне удалось реализовать классический обход по матрице, используя циклы. Это поможет мне легче понимать, как работает язык и облегчит будущую работу с массивами и
матрицами.



\end{document}

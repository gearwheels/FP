\documentclass[12pt]{article}

\usepackage{fullpage}
\usepackage{multicol,multirow}
\usepackage{tabularx}
\usepackage{ulem}
\usepackage[utf8]{inputenc}
\usepackage[russian]{babel}
\usepackage{amsmath}
\usepackage{amssymb}

\usepackage{titlesec}

\titleformat{\section}
  {\normalfont\Large\bfseries}{\thesection.}{0.3em}{}

\titleformat{\subsection}
  {\normalfont\large\bfseries}{\thesubsection.}{0.3em}{}

\titlespacing{\section}{0pt}{*2}{*2}
\titlespacing{\subsection}{0pt}{*1}{*1}
\titlespacing{\subsubsection}{0pt}{*0}{*0}
\usepackage{listings}
\lstloadlanguages{Lisp}
\lstset{extendedchars=false,
	breaklines=true,
	breakatwhitespace=true,
	keepspaces = true,
	tabsize=2
}
\begin{document}

\section*{Отчет по лабораторной работе № 1 \\
по курсу \guillemotleft Функциональное программирование\guillemotright}
\begin{flushright}
Студент группы М8О-307-19 МАИ \textit{Тимофеев Алексей Владимирович}, \textnumero 21 по списку \\
\makebox[7cm]{Контакты: {\tt TImofeevAV8f@yandex.ru} \hfill} \\
\makebox[7cm]{Работа выполнена: 17.03.2022 \hfill} \\
\ \\
Преподаватель: Иванов Дмитрий Анатольевич, доц. каф. 806 \\
\makebox[7cm]{Отчет сдан: \hfill} \\
\makebox[7cm]{Итоговая оценка: \hfill} \\
\makebox[7cm]{Подпись преподавателя: \hfill} \\

\end{flushright}

\section{Тема работы}
Примитивные функции и особые операторы Common Lisp.

\section{Цель работы}
Научиться вводить S-выражения в Lisp-систему, определять переменные и функции, работать с условными операторами, работать с числами, используя схему линейной и древовидной рекурсии.

\section{Задание (вариант № 1.34)}
Поле шахматной доски определяется парой натуральных чисел, каждое из которых не превосходит восьми:\\
первое число - номер вертикали (при счете слева направо).
второе - номер горизонтали (при счете снизу вверх),
Определите на языке Коммон Лисп функцию-предикат с четырьмя параметрами - натуральными числам k, l, m, n, каждое из которых не превосходит восьми.

k, l\\
Задают поле, на котором расположена фигура - ладья.
m, n\\
Задают поле, куда она должен попасть.
Функция должна возвращать

T,\\
если ладья (k,l) может попасть на поле (m,n) за один ход;\\
i, j\\
два значения с помощью values, если ладья (k,l) может попасть на поле (m,n) за два хода через поле (i,j).\\
Примеры\\
(castle-moves 4 4 7 4) => T\\
(castle-moves 1 1 2 7) => 2, 1\\
\section{Оборудование студента}
Процессор Intel Core i5-10600K @ 4.10GHz, память: 16 Gb, разрядность системы: 64.

\section{Программное обеспечение}
ОС Ubuntu 20.04.4 LTS, комилятор GCL (GNU Common Lisp)  2.6.12, текстовый редактор VS Code

\pagebreak
\section{Идея, метод, алгоритм}
Рассмотрим поле шахматной доски, оно определяется парой натуральных чисел.\\
Нужно написать функцию-предикат, которая возвращает T, либо значение поля (i, j).\\
Функция принимает 2 пары аргументов, которые являются координатами 2-х ячеек, первая пара - стартовая ячейка, вторая пара конечная ячейка. Если хотя бы одна координата у этих пар совпадает, это значит, что мы можем передвинуть ладью в конечную точку за один ход, иначе за два хода. Сводим ситуацию к тому, когда мы можем передвинуть за один ход, и печатаем текущие координаты.

\section{Сценарий выполнения работы}

\section{Распечатка программы и её результаты}

\subsection{Исходный код}
\lstinputlisting{./lab1.lisp}

\pagebreak
\subsection{Результаты работы}
\lstinputlisting{./log1.txt}

\pagebreak
\section{Дневник отладки}
\begin{tabular}{|p{50pt}|p{80pt}|p{140pt}|p{140pt}|}
\hline
Дата & Событие & Действие по исправлению & Примечание \\
\hline
\end{tabular}

\section{Замечания автора по существу работы}
Работа показалась мне слишком простой с точки зрения программирования.


\section{Выводы}
Я познакомился с синтаксисом языка Common Lisp. Было непривычно и сложно привыкнуть к способу написания кода, так как много скобок. Также я настроил IDE под Common Lisp и разобрался с компилятором gcl, но в итоге пришел к выводу, что в следующей работе буду использовать CLIPS, потому что он показался мне понятнее и стабильнее в работе.



\end{document}

\documentclass[12pt]{article}

\usepackage{fullpage}
\usepackage{multicol,multirow}
\usepackage{tabularx}
\usepackage{ulem}
\usepackage[utf8]{inputenc}
\usepackage[russian]{babel}
\usepackage{amsmath}
\usepackage{amssymb}

\usepackage{titlesec}

\titleformat{\section}
  {\normalfont\Large\bfseries}{\thesection.}{0.3em}{}

\titleformat{\subsection}
  {\normalfont\large\bfseries}{\thesubsection.}{0.3em}{}

\titlespacing{\section}{0pt}{*2}{*2}
\titlespacing{\subsection}{0pt}{*1}{*1}
\titlespacing{\subsubsection}{0pt}{*0}{*0}
\usepackage{listings}
\lstloadlanguages{Lisp}
\lstset{extendedchars=false,
	breaklines=true,
	breakatwhitespace=true,
	keepspaces = true,
	tabsize=2
}
\begin{document}

\section*{Отчет по лабораторной работе № 5 \\
по курсу \guillemotleft Функциональное программирование\guillemotright}
\begin{flushright}
Студент группы М8О-307-19 МАИ \textit{Тимофеев Алексей Владимирович}, \textnumero 21 по списку \\
\makebox[7cm]{Контакты: {\tt TimofeevAV8f@yandex.ru} \hfill} \\
\makebox[7cm]{Работа выполнена: 29.05.2022 \hfill} \\
\ \\
Преподаватель: Иванов Дмитрий Анатольевич, доц. каф. 806 \\
\makebox[7cm]{Отчет сдан: \hfill} \\
\makebox[7cm]{Итоговая оценка: \hfill} \\
\makebox[7cm]{Подпись преподавателя: \hfill} \\

\end{flushright}

\section{Тема работы}
Обобщённые функции, методы и классы объектов.

\section{Цель работы}
Цель работы: научиться определять простейшие классы, порождать экземпляры классов, считывать и изменять значения слотов, научиться определять обобщённые функции и методы.

\section{Задание (Вариант 5.25)}

Дан экземпляр класса triangle, причем все вершины треугольника могут быть заданы как декартовыми координатами (экземплярами класса cart), так и полярными (экземплярами класса polar).\\

Задание: Определить обычную функцию медиана, возвращающую объект-отрезок (экземпляр класса line), являющийся медианой первого угла vertex1. Концы результирующего отрезка могут быть получены либо в декартовых, либо в полярных координатах.\\

(setq tri (make-instance 'triangle

		\hspace{52pt}:1 (make-instance 'cart-или-polar ...)

		\hspace{52pt}:2 (make-instance 'cart-или-polar ...)

		\hspace{52pt}:3 (make-instance 'cart-или-polar ...)))

(медиана tri) => [ОТРЕЗОК ...]\\


\section{Оборудование студента}
Процессор Intel Core i5-10600K @ 4.10GHz, память: 16 Gb, разрядность системы: 64.

\section{Программное обеспечение}
ОС Ubuntu 20.04.4 LTS, комилятор GNU CLISP 2.49.92, текстовый редактор VS Code

\pagebreak
\section{Идея, метод, алгоритм}
Я взял с сайта нашего курса объявление классов cart, polar, line, triangle, а также метод для
печати каждого класса print-object. Далее написал функцию median работающую только с cart. Так как треугольник в cart обрабатывать я уже умею, было решено сделать функцию, которая преобразует декартовы в полярные координаты.

\section{Сценарий выполнения работы}

\section{Распечатка программы и её результаты}

\subsection{Исходный код}
\lstinputlisting{./lab5.lisp}

\pagebreak
\subsection{Результаты работы}
\lstinputlisting{./log5.txt}

\pagebreak
\section{Дневник отладки}
\begin{tabular}{|p{50pt}|p{80pt}|p{140pt}|p{140pt}|}
\hline
Дата & Событие & Действие по исправлению & Примечание \\
\hline
\end{tabular}

\section{Замечания автора по существу работы}
Программа работает с небольшой погрешностью в методе расчета полярных координат.\\
Возможно это из-за внутренней реализации тригонометрических функций.


\section{Выводы}
При выполнении лабораторной работы № 5 я научился определять простейшие классы, порождать экземпляры классов, считывать и изменять значения слотов,а также научился определять обобщённые функции и методы.




\end{document}

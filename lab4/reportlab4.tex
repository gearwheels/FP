\documentclass[12pt]{article}

\usepackage{fullpage}
\usepackage{multicol,multirow}
\usepackage{tabularx}
\usepackage{ulem}
\usepackage[utf8]{inputenc}
\usepackage[russian]{babel}
\usepackage{amsmath}
\usepackage{amssymb}

\usepackage{titlesec}

\titleformat{\section}
  {\normalfont\Large\bfseries}{\thesection.}{0.3em}{}

\titleformat{\subsection}
  {\normalfont\large\bfseries}{\thesubsection.}{0.3em}{}

\titlespacing{\section}{0pt}{*2}{*2}
\titlespacing{\subsection}{0pt}{*1}{*1}
\titlespacing{\subsubsection}{0pt}{*0}{*0}
\usepackage{listings}
\lstloadlanguages{Lisp}
\lstset{extendedchars=false,
	breaklines=true,
	breakatwhitespace=true,
	keepspaces = true,
	tabsize=2
}
\begin{document}

\section*{Отчет по лабораторной работе № 4 \\
по курсу \guillemotleft Функциональное программирование\guillemotright}
\begin{flushright}
Студент группы М8О-307-19 МАИ \textit{Тимофеев Алексей Владимирович}, \textnumero 21 по списку \\
\makebox[7cm]{Контакты: {\tt TimofeevAV8f@yandex.ru} \hfill} \\
\makebox[7cm]{Работа выполнена: 13.05.2022 \hfill} \\
\ \\
Преподаватель: Иванов Дмитрий Анатольевич, доц. каф. 806 \\
\makebox[7cm]{Отчет сдан: \hfill} \\
\makebox[7cm]{Итоговая оценка: \hfill} \\
\makebox[7cm]{Подпись преподавателя: \hfill} \\

\end{flushright}

\section{Тема работы}
Знаки и строки.

\section{Цель работы}
Цель работы: научиться работать с литерами (знаками) и строками при помощи функций обработки строк и общих функций работы с последовательностями.

\section{Задание (Вариант 2.39)}
Вариант 4.40 (сложность 3)
Запрограммировать на языке Коммон Лисп функцию, принимающую один аргумент - дерево, т.е. список с подсписками, представляющий форму арифметического выражения Лисп. В выражении допустимы только\\
четыре арифметические функции +, -, * и / (предусмотреть случай унарных функций - и /).\\
символы переменных,\\
числовые константы.\\
Функция должна вернуть строку этого арифметического выражения в постфиксной польской записи в предположении, что все арифметические операторы в ней трактуются как бинарные.\\

(form-to-postfix '(+ (* b b) (- (* 4 a c)))) => "b b * 0 4 a c * * - +"\\


\section{Оборудование студента}
Процессор Intel Core i5-10600K @ 4.10GHz, память: 16 Gb, разрядность системы: 64.

\section{Программное обеспечение}
ОС Ubuntu 20.04.4 LTS, комилятор GNU CLISP 2.49.92, текстовый редактор VS Code

\pagebreak
\section{Идея, метод, алгоритм}
Функция получает дерево арифметических выражений Лиспа в виде списка списков. Данный список списков проверяется на несколько случаев:\\
1. Если в подсписке два элемента и первый из них $-$, значит это унарная операция разности.\\
2. Если в подсписке два элемента и первый из них $/$, значит это унарная операция деления.\\
3. Если нет совпадения с двумя верхними случаями, то сразу передаем текущий элемент списка и следующий в функцию f.\\

В функции f с помощью функции g проверяем, является ли элемент атомом, и далее печатаем в виде обратной польской нотации.\\


\section{Сценарий выполнения работы}

\section{Распечатка программы и её результаты}

\subsection{Исходный код}
\lstinputlisting{./lab4.lisp}

\pagebreak
\subsection{Результаты работы}
\lstinputlisting{./log4.txt}

\pagebreak
\section{Дневник отладки}
\begin{tabular}{|p{50pt}|p{80pt}|p{140pt}|p{140pt}|}
\hline
Дата & Событие & Действие по исправлению & Примечание \\
\hline
\end{tabular}

\section{Замечания автора по существу работы}
Довольно интересная задача. Эта задача мне показалась сложнее попавшихся мне ранее. Из-за обилия различных случаев «взаимодействия» операторов с их приоритетами и свойствами.


\section{Выводы}
При выполнении лабораторной работы № 4 я познакомился со списками языка Common Lisp. я научился работать с литерами и строками в языке Коммон Лисп. Написанная программа работает правильно и прошла все тесты.




\end{document}
